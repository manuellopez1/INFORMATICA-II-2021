\documentclass{article}
\usepackage[utf8]{inputenc}
\usepackage[spanish]{babel}
\usepackage{listings}
\usepackage{graphicx}
\graphicspath{ {images/} }
\usepackage{cite}

\begin{document}

\begin{titlepage}
    \begin{center}
        \vspace*{1cm}
            
        \Huge
        \textbf{Desafío: Llevar un objeto de una posición A a una B}
            
        \vspace{0.5cm}
        \LARGE
        
            
        \vspace{1.5cm}
            
        \textbf{Manuel Alejandro López Loaiza}
            
        \vfill
            
        \vspace{0.8cm}
            
        \Large
        Despartamento de Ingeniería Electrónica y Telecomunicaciones\\
        Universidad de Antioquia\\
        Medellín\\
        Marzo de 2021
            
    \end{center}
\end{titlepage}

\tableofcontents

\newpage
\section{Resumen}\label{intro}

El ejercicio tiene como objetivo solucionar un problema con el fin de que la persona lo pueda realizar sin ninguna dificultad, por medio de las instrucciones que se  mostrarán a continuación.

\section{Desarrollo} \label{contenido}

La actividad es la siguiente: Acomodar con una mano dos tarjetas sobre una hoja de papel en forma de pirámide, sin  ayuda alguna.
\vspace{10PT}

Los pasos a seguir son:

\vspace{10PT}

1) Coja la hoja de papel y ubíquela en una superficie plana que esté a no más de 20 centímetros de usted.

\vspace{10PT}

2) Con su mano dominante, ubique las tarjetas en la mitad de la hoja en forma vertical, de manera que queden iguales.

\vspace{10PT}

3) Después procederá a parar las tarjetas verticalmente de manera que las caras de las tarjetas apunten a su derecha e izquierda, apoyando la parte inferior de la tarjetas en la hoja 

\vspace{10PT}

4) Como se observa hay tres lados de las tarjetas que no tienen apoyo, para eso haremos lo siguiente: En los dos lados largos pondremos la yema de los dedos, uno con el pulgar y el otro con el medio después con el tercer lado ubicará su dedo índice haciendo presión para que estas no se caigan.
\vspace{10PT}

5) Ahora con los dedos medio y pulgar se inclinará una de las tarjetas levemente en un ángulo no mayor a 30°, al mismo tiempo con el dedo índice haremos una pequeña presión a la tarjeta que está vertical para que no se mueva

\vspace{10PT}

6) si realizó el paso anterior correctamente puede observar que una tarjeta está levemente inclinada y la otra completamente vertical, para hacer la pirámide deje "caer" suavemente la tarjeta inclinada, todavía sostenida con sus yemas, sobre la hoja tratando de formar la pirámide.

\vspace{10PT}

7) retire lentamente los dedos para no derribar la pirámide 

\vspace{10PT}

8) Si ésta está estable cumplió satisfactoriamente la tarea, si no repita desde el paso (2) hasta completar la tarea 


\section{Conclusión} \label{imagenes}

Para que una labor sea completada de manera correcta y eficaz tiene que ser entendida primero, ahí está la diferencia entre la programación con los seres humanos; ya que la primera sólo tiene una labor, la cual es seguir los pasos que diga un programador en cambio los humanos por más instrucciones que se le designen, pueden llegar a  comprender de  manera diferente, porque el entendimiento siempre cambia dependiendo de cada persona.
 

\end{document}
